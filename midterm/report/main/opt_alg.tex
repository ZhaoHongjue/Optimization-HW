\section{Optimization Algorithms}

\subsection{Gradient Method}

Gradient Method is one of the most common optimization algorithm in many applications, for instance, Deep Learning and so on.
Just as its name, the \emph{antigradient} is choosen to be the search direction of the Gradient Method, since its the locally 
steepest descent of a differentiable function. 

The general algorithm form of Gradient Method is as Algorithm.~\ref{alg:gradient-method}
\begin{algorithm}[!htbp]
    \caption{Gradient Method}\label{alg:gradient-method}
    \KwIn{Objective function \(f(\xB)\), gradient function \(\gB(\xB) = \grad f(\xB)\) and accuracy \(\varepsilon\).}
    \KwOut{Local minimum of \(f(\xB)\): \(\xB^*\).}
    \textbf{Initialization}: Set \(k = 0\) and initialize \(\xB_0 \in \R^n\). \\
    \While{True} {
        Calculate gradient \(\gB_k = \gB(\xB^{(k)})\).\\
        \eIf {\(\norm{\gB_k} < \varepsilon\)} {
            Stop iteration and Let \(\xB^* = \xB_k\).
        }{
            Let \(\xB_{k+1} = \xB_k - h_k \gB_k\)
        }
        \(k = k+1\)
    }
\end{algorithm}

\subsection{Conjugate Gradient Method}

Conjugate gradient methods was initially developed for minimizing quadratic
functions. Consider the problem
\[
    \min_{\xB \in \R^n} f(\xB)  
\]
with \(f(\xB) = \alpha + \innerprod{\aB}{\xB} + \frac{1}{2} \innerprod{\bm{Ax}}{\xB}\)
and \(\bm{A} = \bm{A}^\top \succ 0\). We have known that the solution of this problem is
\(\xB^* = -\bm{A}^{-1}\bm{a}\).

Consider the linear \emph{Krylov} subspaces
\[
    \mathscr{L}_k = \Lin \{ 
        \bm{A} (\xB_0 - \xB^*), \dots, \bm{A}^k (\xB_0 - \xB^*)  
    \}, \quad k \ge 1,  
\]
where \(\bm{A}^k\) is the \(k\)-th power of matrix \(\bm{A}\). A sequence of points \(\{\xB_k\}\)
is generated in accordance with the following rule:
\[
    \xB_k = \mathop{\arg \min} \{ 
        f(\xB) \mid \xB \in \xB_0 + \mathscr{L}_k  
    \}, \quad k \ge 1.  
\]
For any \(k \ge 1\) we have \(\mathscr{L}_k = \Lin \{ \grad f(\xB_0), \dots, \grad f(\xB_{k-1}) \}\). Here
we define \(\bm{\delta}_i = \xB_{i+1} - \xB_i\), and it is also clear that 
\(\mathscr{L}_k = \Lin \{ \bm{\delta}_0, \dots, \bm{\delta}_{k-1} \}\). From this we can derive the algorithm form 
of Conjugate Gradient Method as Algorithm.~\ref{alg:conjugate-grad}.

\begin{algorithm}[!htbp]
    \caption{Conjugate Gradient Method}\label{alg:conjugate-grad}
    \KwIn{Objective function \(f(\xB)\) and accuracy \(\varepsilon\).}
    \KwOut{Local minimum of \(f(\xB)\): \(\xB^*\).}
    \textbf{Initialization}: Set \(k = 0\), and initialize \(\xB_0 \in \R^n\). \\
    Compute \(f(\xB_0)\) and \(\grad f(\xB_0)\) and set \(\pB_0 = \grad f(\xB_0)\).\\
    \While{True} {
        \eIf {\(\norm{\grad f(\xB_k)} < \varepsilon\)} {
            Stop iteration and Let \(\xB^* = \xB_k\).
        }{
            Let \(\xB_{k+1} = \xB_k - h_k \pB_k\). \\
            Compute \(f(\xB_{k+1})\) and \(\grad f(\xB_{k+1})\). \\
            Compute coefficient \(\beta_k\). \\
            Define \(\pB_{k+1} = \grad f(\xB_{k+1}) - \beta_k \pB_k\).
        }
        \(k = k+1\)
    }
\end{algorithm}

There exist many different formulas for the coefficient \(\beta_k\). There are three most popular
expressions.
\begin{enumerate}
    \item {
        Dai-Yuan:
        \[
            \beta_k = \frac{\norm{\grad f(\xB_{k+1})}^2}
            {\innerprod{\grad f(\xB_{k+1}) - \grad f(\xB_{k})}{\pB_k}}  
        \]
    }
    \item {
        Fletcher–Rieves: 
        \[
            \beta_k = -\frac{\norm{\grad f(\xB_{k+1})}^2}{\norm{\grad f(\xB_{k})}^2}  
        \]
    }
    \item {
        Polak–Ribbiere:
        \[
            \beta_k = -\frac{\innerprod{\grad f(\xB_{k+1})}{\grad f(\xB_{k+1}) - \grad f(\xB_{k})}}{\norm{\grad f(\xB_{k})}^2}  
        \]
    }
\end{enumerate}

\subsection{Quasi-Newton Method}

Quasi-Newton Method, which is also called \emph{variable metirc method}, is an alternative of Newton's Method.
The Newton's Method requires Hessian matrix when finding the local minima. Nonetheless, it may be too expensive
to calculate Hessian matrix each iteration. In contrast, quasi-Newton methods can achieve similar performance
even when the Hessian matrix is unavailable.

For Newton's Method, it follows the following equation to finding the local minima:
\[
    \xB_{k+1} = \xB_k - [\hess f(\xB)]^{-1} \grad f(\xB).  
\]
Since the Hessian matrix is unavailable in some cases, the quasi-Newton methods follow the following equation to finding the local minima:
\[
    \xB_{k+1} = \xB_k - h_k \bm{H}_k \grad f(\xB),
\]
in which \(\bm{H}_k \to [\hess f(\xB^*)]^{-1}\). If \(\bm{H}_{k+1}\) satisfies
\[
    \bm{H}_{k+1} = \bm{H}_{k+1}^\top \succ 0 \quad \text{and} \quad 
    \bm{H}_{k+1}(\grad f(\xB_{k+1}) - \grad f(\xB_{k})) = \xB_{k+1} - \xB_k,
\]
we call it satisfies \emph{quasi-Newton rule}. The general algorithm form of quasi-Newton Method is as Algorithm.~\ref{alg:quasi-newton}.

\begin{algorithm}[!htbp]
    \caption{quasi-Newton Method}\label{alg:quasi-newton}
    \KwIn{Objective function \(f(\xB)\) and accuracy \(\varepsilon\).}
    \KwOut{Local minimum of \(f(\xB)\): \(\xB^*\).}
    \textbf{Initialization}: Set \(k = 0\), \(\bm{H}_0 = \bm{I}_n\), and initialize \(\xB_0 \in \R^n\). \\
    Compute \(f(\xB_0)\) and \(\grad f(\xB_0)\).\\
    \While{True} {
        Calculate gradient \(\pB_k = \bm{H}_k \grad f(\xB_k)\).\\
        \eIf {\(\norm{\grad f(\xB_k)} < \varepsilon\)} {
            Stop iteration and Let \(\xB^* = \xB_k\).
        }{
            Let \(\xB_{k+1} = \xB_k - h_k \pB_k\). \\
            Compute \(f(\xB_{k+1})\) and \(\grad f(\xB_{k+1})\). \\
            Update the matrix \(\bm{H}_k\) to \(\bm{H}_{k+1}\).
        }
        \(k = k+1\)
    }
\end{algorithm}

There are several ways to satisfy the quasi-Newton relation. Here we define 
\[
    \Delta \bm{H}_k = \bm{H}_{k+1} - \bm{H}_k, \quad \bm{\gamma}_k = \grad f(\xB_{k+1}) - \grad f(\xB)_k, \quad \bm{\delta}_k = \xB_{k+1} - \xB_k.  
\]

Then three most popular updating methods are as follows:
\begin{enumerate}
    \item {Rank-one correction scheme}:
        \[
            \Delta \bm{H}_k = \frac{(\bm{\delta}_k - \bm{H}_k\bm{\gamma}_k)(\bm{\delta}_k - \bm{H}_k\bm{\gamma}_k)^\top}{\innerprod{\bm{\delta}_k - \bm{H}_k\bm{\gamma}_k}{\bm{\gamma}_k}}  
        \]
    \item {Davidon–Fletcher–Powell scheme (DFP)}:
        \[
            \Delta \bm{H}_k = \frac{\bm{\delta}_k\bm{\delta}_k^\top}{\innerprod{\bm{\gamma}_k}{\bm{\delta}_k}} 
            - \frac{\bm{H}_k\bm{\gamma}_k\bm{\gamma}_k^\top\bm{H}_k}{\innerprod{\bm{H}_k\bm{\gamma}_k}{\bm{\gamma}_k}}
        \]
    \item {Broyden–Fletcher–Goldfarb–Shanno scheme (BFGS)}:
        \[
            \Delta \bm{H}_k = \beta_k \frac{\bm{\delta}_k\bm{\delta}_k^\top}{\innerprod{\bm{\gamma}_k}{\bm{\delta}_k}} 
            - \frac{\bm{H}_k\bm{\gamma}_k\bm{\delta}_k^\top + \bm{\delta}_k\bm{\gamma}_k^\top\bm{H}_k}{\innerprod{\bm{\gamma}_k}{\bm{\delta}_k}}
        \]
        where \(\beta_k = 1 + \innerprod{\bm{H}_k\bm{\gamma}_k}{\bm{\gamma}_k} / \innerprod{\bm{\gamma}_k}{\bm{\delta}_k}\).
\end{enumerate}

\subsection{Step-Length Selection}

Gradient Method, Conjugate Gradient Method and Quasi-Newton Method are all \emph{the first-order optimization methods}, 
which can be generally expressed as
\[
    \xB_{k+1} = \xB_k + h_k \pB_k,  
\] 
in which \(\pB_k\) is the \emph{search direction} which is related to \(\grad f(\xB_k)\) and \(h_k\) is the step size. For the step size, we usually hope
it satisfies
\[
    h_k = \min_{h} f(\xB_k + h_k \pB_k).  
\]
When the objective function is quadratic, we can derive \(h_k\) analytically.

\begin{thm}\label{thm:step-minimizer}
    If \(f\) is convex quadratic
    \[
        f(\xB) = \frac{1}{2} \xB^\top \bm{Q} \xB - \bB^\top \xB,  
    \]
    Its one-dimensional minimizer along \(\xB_k + h_k \pB_k\) is given by
    \[
        h_k = -\frac{\grad f(\xB_k)^\top \pB_k}
        {\pB_k^\top \bm{Q}\pB_k}    
    \]
\end{thm}

The proof of Theorem.~\ref{thm:step-minimizer} can be found in Appendix.~\ref{app:proof}.